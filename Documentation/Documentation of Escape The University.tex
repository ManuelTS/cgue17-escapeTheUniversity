\documentclass[12pt]{article}
\usepackage[english]{babel}
\usepackage{geometry}
\usepackage{amssymb}
\usepackage{acronym}
\usepackage{graphicx}
\usepackage{listings}
\usepackage{color}
\usepackage{verbatim}
\usepackage{multicol}
\usepackage{wrapfig}
\usepackage[colorlinks,pdfpagelabels,pdfstartview = FitH,bookmarksopen = true,bookmarksnumbered = true, ,plainpages = false,hypertexnames = false, citecolor=blue,filecolor=blue,linkcolor=blue,urlcolor=blue] {hyperref}


\geometry{a4paper,left=2cm,right=2cm, top=2cm, bottom=2cm}

\definecolor{sh_comment}{rgb}{0.12, 0.38, 0.18 } %adjusted, in Eclipse: {0.25, 0.42, 0.30 } = #3F6A4D
\definecolor{sh_keyword}{rgb}{0.37, 0.08, 0.25}  % #5F1441
\definecolor{sh_string}{rgb}{0.06, 0.10, 0.98} % #101AF9

\def\lstsmallmath{\leavevmode\ifmmode \scriptstyle \else  \fi}
\def\lstsmallmathend{\leavevmode\ifmmode  \else  \fi}

\lstset {
 language=Java,
 frame=shadowbox,
 rulesepcolor=\color{black},
 showspaces=false,showtabs=false,tabsize=2,
 numberstyle=\tiny,numbers=left,
 basicstyle= \footnotesize\ttfamily,
 stringstyle=\color{sh_string},
 keywordstyle = \color{sh_keyword}\bfseries,
 commentstyle=\color{sh_comment}\itshape,
 captionpos=b,
 xleftmargin=0.7cm, xrightmargin=0.5cm,
 lineskip=-0.3em,
 escapebegin={\lstsmallmath}, escapeend={\lstsmallmathend}
}

\setlength{\parindent}{0pt} % Kein Einzug
\begin{document}
\begin{titlepage}
 \vspace*{8cm}
\begin{center}
 \textbf{\Huge Escape the University} \\
  \vspace{3mm}
 {\Large Submission 1 Documentation Document\\
  \vspace{3mm}
  \today}
 \vspace{5mm}
\begin{table}[h!]
  \centering
  \begin{tabular}{c c}
	\textbf{Manuel T. Schrempf}  &   \textbf{Stefan Wilker}\\
	\href{mailto:e0920136@student.tuwien.ac.at}{e0920136@student.tuwien.ac.at}&
	\href{mailto:e0920293@student.tuwien.ac.at}{e0920293@student.tuwien.ac.at}\\
  \end{tabular}
\end{table}

%titlehead{\centering\includegraphics[width=6cm]{escapetheuniversity_poster_a4.png}}
 \vspace{3mm}
{\large Study program: 066 935 Medieninformatik\\
 \vspace{2mm}
Class: 186.831 UE Computergraphics}
\end{center}
\end{titlepage}
\tableofcontents % Erzeugt Inhaltsangabe
%\listoffigures
%\listoftables
%\lstlistoflistings
\newpage
\pagestyle{plain}
\setcounter{page}{1}

\section{Controls \label{Controls}}

\begin{table}[h!]
  \centering
  \label{table1}
  \begin{tabular}{p{3cm} c c}
Key & Function\\
    \hline
W & Walk forwards	\\
A & Strafe left \\
S & Walk backwards \\
D & Strafe right \\
E & Open doors\\
Q & Open the door with the ``enemy'' behind.\\

Mouse & Look around\\

 ESC or END & Quit game \\
 F3 & Wireframe mode on or off \\
 F6 & Depth visualization on or off\\ % https://learnopengl.com/#!Advanced-OpenGL/Depth-testing
 F7 & Pause or Resume game \\
  \end{tabular}
\end{table}

\section{Development Status}
\subsection{Camera}
The free moving camera is implemented as on the \href{http://www.learnopengl.com/#!Getting-started/Camera}{tutorial on camera from learnOpenGL.com} with slight modification to suit our needs, e.g. scroll speed, include all depth values on movement etc.

\subsection{Moving Objects}
\begin{itemize}
\item Opening and closing doors.
\end{itemize}

\subsection{Texture Mapping}
See section \ref{Texture Mapping}.
\subsection{Simple Lighting and Materials}
See section \ref{lightSources}.
\subsection{Controls}
See section \ref{Controls}.
\subsection{Basic Gameplay}
You find yourself in the middle of the university in the inner courtyard. You need to get out of the building without beeing detected. Open the doors and find yourself a way towards the paved floor. Do not open the wrong door, otherwise you encounter a professor and you are busted!
This is just a trial run... consider yourself planning your escape for the real experience that will follow. Explore the building, get the floor plane into your head. It will help your real escape in the near future.


\section{Effects}
\begin{itemize}
	\item Blinn Phong lighting with materials from \href{http://learnopengl.com/#!Advanced-Lighting/Advanced-Lighting}{learnopengl.com advanced lighting tutorial}.
\end{itemize}

\section{Features}
\begin{itemize}
	\item Temporary godlike powers: You can open all doors at once! Except the ``bad'' one.
	\item Feel free to explore the floor plane of the building without collision and enjoy a beautifully designed building.
	\item Usage of UBOs \cite{openGLSuperBible}.
	%\item Gamma correction \cite{openGLSuperBible}.
	\item Deferred shading from \cite{openGLSuperBible} and \href{http://learnopengl.com/#!Advanced-Lighting/Deferred-Shading}{learnopengl.com deferred shaing tutorial} with light volumes form
	\href{http://ogldev.atspace.co.uk/www/tutorial36/tutorial36.html}{tutorial number 36} and \href{http://ogldev.atspace.co.uk/www/tutorial37/tutorial37.html}{37} from ogldev.atspace.co.uk.
	\item Text rendering on screen with signed distance fields as in \cite{signedDistanceFields}.
	% Original code: https://github.com/lazarmitic/SDFTR-GL
	% Rendering: https://en.wikibooks.org/wiki/OpenGL_Programming/Modern_OpenGL_Tutorial_Text_Rendering_01
	% Optimization: https://en.wikibooks.org/wiki/OpenGL_Programming/Modern_OpenGL_Tutorial_Text_Rendering_02#Rendering_lines_of_text_using_the_atlas
	% Calculation of SDF: https://github.com/libgdx/libgdx/wiki/Distance-field-fonts
	\item Scenegraph and different node type usage.
  \item \href{https://www.youtube.com/watch?v=2LW9JSYn_h0}{Radar} frustum culling from a \href{http://www.lighthouse3d.com/tutorials/view-frustum-culling/radar-approach-implementation-ii}{lighthouse3d.com tutorial} with a lot own changes.
  % https://stackoverflow.com/questions/8101119/how-do-i-methodically-choose-the-near-clip-plane-distance-for-a-perspective-proj#8101234
	% https://stackoverflow.com/questions/13896385/calculate-near-and-far-value-for-glperspective-in-opengl#13896488
  \item Physics done with \href{http://bulletphysics.org/wordpress/}{bullet}.
  % http://bulletphysics.org/Bullet/BulletFull/
\end{itemize}

%Brief description of the implementation, in particular a short description of how the different aspects of the requirements (see above) were implemented.

\section{Textured objects \label{Texture Mapping}}
\begin{itemize}
\item Door + door handle + doorframe
\item Hanging lamp
\item Table + chairs for outside use
\item Male character
\item Whole building
\item Paved pathwalk outside the building
\end{itemize}

\section{Light Sources \label{lightSources}}
All objects are now illuminated with blinn phong lighting from a \href{http://learnopengl.com/#!Advanced-Lighting/Advanced-Lighting}{blinn phong tutorial on learnOpenGL.com}. Deferred shading form \cite{openGLSuperBible}, \href{http://learnopengl.com/#!Advanced-Lighting/Deferred-Shading}{learnopengl.com deferred shaing tutorial}, and \href{http://www.lighthouse3d.com/tutorials/opengl_framebuffer_objects/}{lighthouse3d.com openGL framebuffer objects}.\\

Directional and point lights done with the \href{http://learnopengl.com/#!Lighting/Light-casters}{light casters tutorial on learnOpenGL.com}.\\


The illuminated light source is Vorlesungssaal\_light.

\section{Additional libraries in use}
 %What additional libraries (eg for collision, object-loader, sound, …) were used, including references (URL) (see restrictions)?

\begin{itemize}
\item GLFW from \url{http://www.glfw.org/}
\item GLEW from \url{http://glew.sourceforge.net/}
\item DevIL from \url{http://openil.sourceforge.net/}
\item irrKlang from \url{http://www.ambiera.com/irrklang/index.html}
\item GLM from \url{http://glm.g-truc.net/}
\item Assimp from \url{http://www.assimp.org/}
\end{itemize}


\section{External resources}

\subsection{Textures and blender models}

\begin{itemize}

\item If not explicit declared otherwise, textures have been taken from
\url{https://lva.cg.tuwien.ac.at/textures/} or painted by ourself (single color textures mostly).


\item The garden table model as well as its textures have been taken from the freely available download-package (after sign-up with email) from the website
\url{http://www.chocofur.com/}. It has been modified in order to have the correct display behaviour in OpenGL as well as a reduced triangle load.

\item The textures for the doors \& door frames are also from the previously mentioned download package. The model was designed by ourselves.

%\item Male Character Model was taken from \url{http://tf3dm.com/3d-model/generic-male-02-81493.html}, textures included.

\item White roughcast texture for Walls from \url{https://freestocktextures.com/texture/seamless-roughcast-wall,812.html}.

\item Ceiling light from \url{https://www.cgtrader.com/free-3d-models/architectural-details/lighting/ceiling-lamp-interior}.

\item Key from \url{https://www.cgtrader.com/free-3d-models/household/other/worn-key} in respect to editorial / non-commercial license.

\item Bookshelf from \url{https://www.cgtrader.com/free-3d-models/furniture/cabinets-storage/bookshelf} in respect to editorial / non-commercial license.

\item University chair from \url{https://www.cgtrader.com/free-3d-models/furniture-set/other/university-portfolio} in respect to the general license. Model has been modified and is therefore not distributed in the way it has been downloaded.

\item Table from \url{https://www.cgtrader.com/free-3d-models/architectural-details/decoration/wood-table-13cac5a4-f11f-4b36-9a77-af84a5f4c914} in respect to the general license. Model has been modified and is therefore not distributed in the way it has been downloaded.

\item Flowerpot from \url{http://archive3d.net/?a=download\&id=e9325e5e#}. In agreement to point ``4. This model may be freely modificated or elaborated."" of Archive3D.net

\item Woman from \url{https://www.cgtrader.com/free-3d-models/character/woman/female-3dmodel} in agreement to custom license, since the description allows ``you may use it for your game or project etc``.

\item School bench from
\url{ https://www.cgtrader.com/3d-models/interior/office/school-desk-206e3c48-1bea-40d0-9c75-cbb46eaa64af}. Model has been purchased and is used in compliance with the Royalty Free License.

\item Freddie Krueger like character from
\url{https://www.cgtrader.com/free-3d-models/character/man/sketchfast-7-halloween-contest}. Model is used in compliance with the Editorial License.

\end{itemize}

\subsection{Sounds}

\begin{itemize}
\item \href{http://freemusicarchive.org/music/EAT/20100129104001364/Oh_Mom_Elevator_Mix}{EAT - Oh Mom Elevator Mix} with an \href{http://creativecommons.org/licenses/by-sa/3.0/us/}{Attribution-Share Alike 3.0 United States License}
\item \href{http://freemusicarchive.org/music/TMHECTOR/The_Haunted_Mansion/DS10Forumcom\_-\_DS10Forumcom\_-\_The_Haunted_Mansion\_-\_16\_The\_Elevator}{TMHECTOR\_-\_16\_-\_The\_Elevator} with an \href{http://creativecommons.org/licenses/by/3.0/}{Creative Commons Attribution 3.0 License}

\item \href{http://freemusicarchive.org/music/Jahzzar/Smoke_Factory/The_last_ones}{The last ones by Jahzzar} with an \href{http://creativecommons.org/licenses/by-sa/3.0/}{Attribution-ShareAlike 3.0 International License.}

% The door opening sounds have been taken from \url{http://www.freesfx.co.uk} in respect to their license of use.
% For stefan: please state the license
\end{itemize}

\section{Possible Optimizations}
Implement and optimization of deferred rendering, this can be for one point \href{deferred illumination}{https://www.opengl.org/discussion_boards/showthread.php/180940-Deferred-shading-and-light-volumes}
or \href{http://ogldev.atspace.co.uk/www/tutorial37/tutorial37.html}{this}, or for two points \textit{tile based deferred rendering}.

\bibliographystyle{plain}
\bibliography{Referenzen}

\end{document}
